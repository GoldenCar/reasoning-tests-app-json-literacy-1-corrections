\documentclass{article}
\usepackage[utf8]{inputenc}
\usepackage{amsmath}

\title{Numerical Reasoning Test Solutions Test 7}
\author{Jaklyn Crilly}
\date{}

\begin{document}

\maketitle

$\textbf{Question 1} \\$
There was a total number of 54700 candidates in June 2007, and 3$\%$ of these candidates were unclassified. The number of unclassified candidates, denoted $C_U$, is then:
\begin{align*}
C_U &= 54700 \times \frac{3}{100}\\
&=1641.
\end{align*}
So there are a total of 1641 unclassified candidates in June 2007. $\\$

$\textbf{Question 2} \\$
The percentage of candidates that achieved a grade of A, B, or C, denoted $P_{ABC}$, is given by the sum of the individual percentages of candidates who achieved a grade of A, a grade of B, or a grade of C. That is:
\begin{align*}
P_{ABC} &= 44 + 21 + 16\\
&= 81.
\end{align*}
Therefore, 81$\%$ of candidates achieved a grade from A-C. $\\$

Given there is a total of 54700 candidates, the total number of candidates that achieved a grade from A-C, denoted $N_{ABC}$, is:
\begin{align*}
N_{ABC} &= 54700 \times \frac{81}{100}\\
&= 44307.
\end{align*}
The answer is therefore 44307. $\\$

$\textbf{Question 3} \\$
Given 44$\%$ of the total 54700 candidates achieved a grade of A, the total number of candidates that achieved this grade, denoted $N_A$, is:
\begin{align*}
N_A  &= 54700 \times \frac{44}{100}\\
&= 24068.
\end{align*}
Given 12039 of these 24068 candidates were female, the number of males that achieved a grade of A, denoted $N_M$, is:
\begin{align*}
N_M &= 24068-12039\\
&=12029.
\end{align*}
So the answer is 12029. $\\$

$\textbf{Question 4} \\$
We know that 22$\%$ of the students who achieved a grade  of below a C in  A-level mathematics re-sat it the following year. Of this 22$\%$, 60$\%$ improved their grade, and we know that 60$\%$ of 22 is:
$$22 \times \frac{60}{100} = 13.2.$$
So of the people who achieved a grade of less than C, 13.2$\%$ of those people re-sat A-level mathematics $\textbf{and}$ improved their grade. $\\$

$\textbf{Question 5} \\$
In the UK in year 1, the GDP was $\$2154 \times 10^{3}$ million, and the population was 60 million. The GDP per head of population, denoted $\text{GDP/P}$, is then given by the GDP divided by the population:
\begin{align*}
\text{GDP/P} &= \frac{2154 \times 10^3}{60}\\
&= 35900,
\end{align*}
(where the value in the numerator and denominator were both in units of millions and so should have both been multiplied by a factor of $10^6$. These factors would cancel each other however, as $10^6/10^6=1$, and so can be ignored).$\\$

The GDP per head of population is thus $\$$35900. $\\$

$\textbf{Question 6} \\$
Need to calculate the GDP per head of population for the five countries in year 1, and determine which is the largest. $\\$

The units millions can be ignored for this question as was establish in the previous question (although note, we must not include units of millions for both the GDP and the population for this to work, not just one of the values). The factor of 1000 can also be ignored as it will affect the value of the 5 countries equally. This question thus reduces down to working out what the largest value of the first column divided by the second column is, so letting $\text{GDP/P}_{\text{country}}$ denote this value for the given country, we get:
\begin{align*}
\text{GDP/P}_{\text{China}} &= \frac{1970}{1572} = 1.25...\\
\text{GDP/P}_{\text{Japan}} &= \frac{4890}{133} = 36.76...\\
\text{GDP/P}_{\text{Netherlands}} &= \frac{599}{18} = 33.27...\\
\text{GDP/P}_{\text{USA}} &= \frac{11998}{296} = 40.53...\\
\text{GDP/P}_{\text{UK}} &= \frac{2154}{60} = 35.9.\\
\end{align*}
The largest GDP per head of population occurs in the USA. $\\$

$\textbf{Question 7} \\$
The GDP per head of population in the Netherlands in year 2, denoted $(\text{GDP/P})_2$, is given by:
\begin{align*}
(\text{GDP/P})_2 &= \frac{610 \times 10^3}{20}\\
&= 30500.
\end{align*}

Now we want to determine the GDP per head of population after the increase in GDP and population. 
If the GDP increased by 5$\%$ in year 2, the new GDP which we will denote by $GDP_N$, is:
\begin{align*}
\text{GDP}_N&= 610 \times \frac{100 + 5}{100} \\
&= 640.5.
\end{align*}

The population increased by 2$\%$ in year 2, and so the new population, denoted $P_N$, is:
\begin{align*}
P_N &= 20 \times 1.02\\
&= 20.4.
\end{align*}

The new GDP per head of population, $\text{GDP/P}_N$, is then given by the new GDP divided by the new population, and is thus:
\begin{align*}
(\text{GDP/P})_N &= \frac{\text{GDP}_N}{P_N}\\
&= \frac{640500}{20.4}\\
&= 31397.0588...
\end{align*}

Lastly, the percentage increase of the GDP per head of population, denoted $P_{I}$, is then given by the difference between the new GDP per head of population and the initial GDP per head of population, as a percentage over the initial GDP per head of population in year 2. That is:
\begin{align*}
P_{I} &= \frac{(\text{GDP/P})_N - (\text{GDP/P})_2}{(\text{GDP/P})_2} \times 100\\
&= \frac{31397.06-30500}{30500}\times 100\\
&= 2.94...
\end{align*}
The answer to one decimal place  is thus 2.9$\%$. $\\$

$\textbf{Question 8} \\$
Although Blackwood and Greenways both have the highest percentage of Finance staff (with Finance accommodating for 12$\%$ of their total staff), the total number of staff at each site is not known so it is not possible to say which site has the greatest number of Finance staff. $\\$

As an example, Blackwood could have 1000 staff in total and Greenways 100, in which case Blackwood would have 120 Finance staff and Greenways only 12. On the other hand, Redbrook could have 10000 staff which would mean they have 1000 Finance staff and hence, Redbrook would have the most. Thus, its is clear that without details on the number of staff, the percentages tell us nothing about which company has the most Finance staff. $\\$

$\textbf{Question 9} \\$
Given 50$\%$ of the staff at the Redbrook site were in Production, by reducing the number of Production staff by 50$\%$ and hence, by half, this means 25$\%$ of the total staff will remain, whilst 25$\%$ will leave. So of the original 100$\%$, only 75$\%$ of staff in total remain. $\\$

If we rebase this by dividing all percentages for Redbrook by 75 and multiplying by 100 to get it back in percentage form, the previous 75$\%$ total is now expressed as the new 100$\%$. The percentage of Redbrook staff that are now employed in Design (given the original percentage was 6$\%$) is thus:
$$\frac{6}{75} \times 100 = 8.$$
So after reducing the Production staff by 50$\%$, 8$\%$ of the remaining staff were in Design. $\\$

$\textbf{Question 10} \\$
We have been given no information in regards to the total number of staff at each site, and so it is impossible to calculate staff sizes or say what percentage of staff there is in Technology. $\\$

Note: Percentages of different total figures cannot be averaged or added together as they are not related.$\\$

$\textbf{Question 11} \\$
The number produced increased by 0.5$\%$ in October and decreased by 0.5$\%$ in November. This does not mean (as we we will soon see) that the level of production stayed the same. This is because the percentages are applied to different figures; that is, the increase of 5$\%$ is applied to a different value than the decrease of 5$\%$ is. $\\$

The number produced in October, denoted $N_O$, is 0.5$\%$ more than 440000, and is thus:
\begin{align*}
N_O &= 440000 + 440000 \times \frac{0.5}{100}\\
&= 442200.
\end{align*}
The number produced in November, denoted $N_N$, is 0.5$\%$ less than 442200, and is thus:
\begin{align*}
N_O &= 442200 - 442200 \times \frac{0.5}{100}\\
&= 439989.
\end{align*}
Therefore, 439989 TP-01s were produced in November 2009. $\\$

$\textbf{Question 12} \\$
Given we do not know the actual volumes produced of these products (only the percentage changes from month to month), we cannot say which volume was the highest. $\\$

$\textbf{Question 13} \\$
In order to determine the percentage change from August 2009 to November 2009, we will compound the individual month to month percentage increases. Letting the percentage change for a specified month and product be denoted by $C_{\text{product}}^{\text{month}}$, then the percentage change for a specified product over this period, denoted by $P_{pr}$, is:
\begin{align*}
P_{pr} &= (100+C_{pr}^{\text{Sep}}) \bigg(\frac{100+C_{pr}^{\text{Oct}}}{100} \bigg) \bigg(\frac{100+C_{pr}^{\text{Nov}}}{100}\bigg)-100.
\end{align*}

Applying this formula to the four products we get:
\begin{align*}
P_{TP-02} &= \bigg(101.2 \times \frac{100.2}{100} \times \frac{99.7}{100} \bigg)-100\\
&= 1.09...\\
P_{SEC-60} &= \bigg( 100.1 \times \frac{100}{100} \times \frac{100}{100}\bigg) -100\\
&=0.1,\\
P_{SEC-130} &= \bigg( 103 \times \frac{99.7}{100} \times \frac{100}{100}\bigg)-100\\
&=2.691,\\
P_{MIN-33} &= \bigg(102 \times \frac{99.9}{100} \times \frac{100.2}{100}\bigg)-100\\
&=2.10...
\end{align*}
The biggest percentage change occurs for the product SEC-130 with a percentage change of 2.691$\%$. $\\$

$\textbf{Question 14} \\$
From the data we have been given, to answer this question we would need to know the number of TP-01 produced in December 2009 (or any month as we could then use the percentage changes to work out the number for December 2009). We would then need to know the number of TP-02 produced in November so we can determine by what percentage we need to increase production by in order to equal that of TP-01. $\\$

We have not been given the numbers of TP-01 or TP-02 produced and so it is impossible to answer this question based just on the month to month percentage changes. $\\$

$\textbf{Question 15} \\$
To determine the total revenue for a given month, denoted $R_{\text{month}}$, we must sum together the revenue for food, clothing and other, for the given month. Doing this for the five possible answers we get:
\begin{align*}
R_{\text{Jan}} &= 12.3+7.1+2.1=21.5,\\
R_{\text{Feb}} &= 12.5+6.9+2.2 = 21.6,\\
R_{\text{Mar}} &= 12.7+8.0+2.0 = 22.7,\\
R_{\text{Apr}} &= 12.2+7.3+1.9 = 21.4,\\
R_{\text{May}} &= 12.0+8.7+2.3 = 23.0.
\end{align*}

The highest total revenue was made in May with a total revenue of £23 million. $\\$

$\textbf{Question 16} \\$
We know that the total revenue made from January to March, denoted $R_T$ is the sum of the total revenue of the months January, February and March. Using our results from the previous question, we get:
\begin{align*}
R_T &= 21.5+21.6+22.7\\
&= 65.8.
\end{align*}

The total revenue due to clothing made from January to March, denoted $R_C$, is:
\begin{align*}
R_C &= 2.1+2.2+2.0\\
&= 6.3.
\end{align*}

The revenue from clothing over this period written as a percentage of the total revenue over this period, which we denote by $P_C$, is:
\begin{align*}
P_C &= \frac{6.3}{65.8} \times 100\\
&= 9.5744...
\end{align*}
To one decimal point, we get that the answer is 9.6$\%$. $\\$

$\textbf{Question 17} \\$
We want to work out when the revenue of clothing for a month is more than one tenth of the total months revenue. So we will divide the total revenue of each month by 10, and then every month for which the clothing revenue is more than this value, we will plus one month to the answer. So one tenth of the total revenue for each month is: 
\begin{align*}
\text{Jan} &= \frac{12.3+7.1+2.1}{10}=2.15,\\
\text{Feb} &= \frac{12.5+6.9+2.2}{10}=2.16,\\
\text{Mar} &= \frac{12.7+8.0+2.0}{10}=2.27,\\
\text{Apr} &= \frac{12.2+7.3+1.9}{10}=2.14,\\
\text{May} &= \frac{12.0+8.7+2.3}{10}=2.30,\\
\text{Jun} &= \frac{11.8+9.0+1.8}{10}=2.26.
\end{align*}
Comparing these to the revenue for clothing for the given month, we see that the revenue of clothing is more than one tenth of the total month's revenue for the month of February, and no others. The answer is therefore 1. $\\$

Note that in May the revenue for clothing is equal to one tenth of the total revenue. Given the question specifically asks for which months clothing's revenue is $\textbf{more}$ than a tenth of the total revenue, May does not count in the solution. $\\$

$\textbf{Question 18} \\$
We know that €1 is worth 1.439277814 CHF, as given in the table. That is:
$$\text{€}1=1.439277814 \, \text{CHF}.$$
Multiplying both sides of this equation by 210, we find:
\begin{align*}
\text{€}210 &= 1.439277814 \times 210 \, \text{CHF}\\
&= 302.2483...\, CHF.
\end{align*}
So to two decimal places, we get that €210 is equivalent to 302.25 CHF. $\\$

$\textbf{Question 19} \\$
We know that €1 is worth £0.892590676. That is:
\begin{align*}
\text{£}0.892590676 &= \text{€}1.
\end{align*}
If we now divide both sides of this equation by 0.892590676, we get:
\begin{align*}
£1 &= \text{€}\frac{1}{0.892590676}\\
&= \text{€}1.1203...
\end{align*}
So we see that to two decimal places, £1 is worth €1.12. $\\$

$\textbf{Question 20} \\$
From the table, we know that 
$$\$1.360493561 = \text{€}1. $$
Multiplying both sides of this equation by 20, and dividing by 1.360493561, we get:
\begin{align*}
\$20 &= \text{€}1 \times \frac{20}{1.360493561}\\
&= \text{€}14.700547340.
\end{align*}

Now we also know that 1€ is worth 25.40084672 Koruny. So we get the equation:
$$ \text{€}1 = 25.40084672 \text{ Koruny}.$$
Multiplying both sides of the equation by $14.700547340$, we get:
\begin{align*}
\text{€}14.700547340 &= 25.40084672 \times 14.700547340 \text{ Koruny}\\
&= 373.406349498 \text{ Koruny}.\\
\end{align*}
Therefore, to two decimal places, we get that $\$$20 is worth €14.70, which in turn, is worth 373.41 Koruny. $\\$

$\textbf{Question 21} \\$
We know that €1 is worth 7.997178477 NOK, and so we get
$$1 \text{ NOK} = \text{€}\frac{1}{7.997178477}.$$
If the value of the Norway Kroner rose by 0.7$\%$, then we get:
\begin{align*}
1 \text{ NOK} &= \text{€}\frac{1}{7.997178477} \times 1.007\\
&= \text{€}0.125919411.
\end{align*}
Now multiplying both sides of this equation by 16:
\begin{align*}
16 \text{ NOK} &= \text{€}0.125919411 \times 16\\
&= \text{€}2.014710569.
\end{align*}

Applying the same procedure to the Denmark Kroner,we know that €1 is worth 7.440844142 DKK, and so we get:
$$1 \text{ DKK} = \text{€}\frac{1}{7.440844142}.$$
If the value of the Denmark Kroner fell by 0.2$\%$, then after this fall:
\begin{align*}
1 \text{ DKK} &= \text{€}\frac{1}{7.440844142} \times 0.998\\
&= \text{€}0.134124567.
\end{align*}
Now dividing both sides of the equation by 0.134124567 and then multiplying by 2.014710569 we find
\begin{align*}
\text{€}2.014710569 &= 1 \times \frac{2.014710569}{0.134124567} \text{ DKK}\\
&= 15.021189712 \text{ DKK}.
\end{align*}
So we see that to two decimal places, 16 NOK is equivalent to €2.01, which in turn is equivalent to 15.02 DKK. 
\end{document}



