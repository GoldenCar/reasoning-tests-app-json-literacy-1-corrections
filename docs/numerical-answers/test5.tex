\documentclass{article}
\usepackage[utf8]{inputenc}
\usepackage{amsmath}

\title{Numerical Reasoning Test Solutions Test 5}
\author{Jaklyn Crilly}
\date{}

\begin{document}

\maketitle

$\textbf{Question 1} \\$
The average price of the iPhone was £560 in 2007 Q4, whilst the competitor phone was £430. Given the prices each increased by $7\%$ over the next quarter, the price of the iPhone in 2008 Q1, denoted $P_I$, and the price of the competitor phone in 2008 Q1, denoted $P_C$, are:
\begin{align*}
P_I &= 560 + 560 \times \frac{7}{100}\\
&= 599.2,\\
P_C &= 430 + 430 \times \frac{7}{100}\\
&= 460.1.
\end{align*}
Therefore, the price of the iPhone in 2008 is £599.2 and the price of the competitor phone is £460.1.$\\$

The iPhone had $2.315 \times 10^6$ sales in the first quarter of 2008, so given the above price of the iPhone in this quarter, the total revenue was $£599.2 \times 2.315 \times 10^6 = £1387.148 \times 10^6$. Similarly for the competitor phone, we find the competitor phone's revenue was $£460.1 \times 1.180 \times 10^6 = £542.918 \times 10^6$. $\\$

The difference in revenue, $D$, is then given by the iPhone's revenue minus the competitor phone's revenue giving:
\begin{align*}
D &= (1387.148 - 542.918) \times 10^6\\
&= 844.23 \times 10^6.
\end{align*}
The solution is thus $£844,230,000$. $\\$

$\textbf{Question 2} \\$
We are not given enough information. The average price of the handset cannot be used to determine the total revenue, as the actual prices of the handsets for each quarter is needed. $\\$

$\textbf{Question 3} \\$
For each of the quarters, we must divide the competitors sales by the iPhone's sales, and then determine which quarter has the smallest value. Calculating to three decimal places, we get:
\begin{align*}
\text{2007 Q3} &: \frac{0.101}{0.270}= 0.374\\
\text{2007 Q4} &: \frac{0.247}{1.119}= 0.221\\
\text{2008 Q1} &: \frac{1.180}{2.315}= 0.510\\
\text{2008 Q2} &: \frac{1.650}{1.703}= 0.969\\
\text{2008 Q3} &: \frac{1.080}{0.710}= 1.521\\
\text{2008 Q4} &: \frac{3.212}{6.890}= 0.466
\end{align*}
The smallest value is 0.221 which occurs in 2007 Q4. $\\$

$\textbf{Question 4} \\$
First we need to calculate the the sales for each book in each region, and so given we know the total sales worldwide of each book we just need to multiply this by the sales percentage for the given region to obtain these values. We then have to minus the book sales of Alice in Wonderland against the book sales of Angela's Ashes and find out which region produces the largest value. The difference in sales between Angela's Ashes and Alice in Wonderland is thus:
\begin{align*}
\text{Australia} &: (0.10 \times 2.2) - (0.15 \times 2.7) = -0.185,\\
\text{Europe} &: (0.22 \times 2.2) - (0.15 \times 2.7) =0.079,\\
\text{USA} &: (0.27 \times 2.2) - (0.25 \times 2.7) = -0.081,\\
\text{Japan} &: (0.12 \times 2.2) - (0.17 \times 2.7) = -0.195,\\
\text{RoW} &: (0.29 \times 2.2) - (0.28 \times 2.7) = -0.118.
\end{align*}

The only positive sales difference and so the only region in which Angela's Ashes had more sales than Alice in Wonderland was in Europe. Hence the answer is Europe. $\\$

$\textbf{Question 5} \\$
Calculating the USA and European sales for Alice in Wonderland we get:
\begin{align*}
\text{USA} &: \frac{25}{100} \times £2.7m = £0.675m\\
\text{Europe} &: \frac{15}{100} \times £2.7m = £0.405m
\end{align*}
The USA sales remain constant and thus will remain at £0.675m each year. The European sales however increase $15\%$ each year and so the interest will be compounded. We want to establish how many years it will take for the European sales to overtake the USA sales, and we will do this year by year until we have overtaken £0.675. So the sales at the end of the following year (given to three decimal places) are:
\begin{align*}
\text{Year }1 &: 0.405 + ( 0.405 \times 0.15 ) = 0.466\\
\text{Year }2 &: 0.466 + ( 0.466 \times 0.15 ) = 0.536\\
\text{Year }3 &: 0.536 + ( 0.536 \times 0.15 ) = 0.616\\
\text{Year }4 &: 0.616 + ( 0.616 \times 0.15 ) = 0.708
\end{align*}
Therefore it takes four years for the European sales of the book Alice in Wonderland to overtake the USA sales of £0.675m. $\\$

$\textbf{Question 6} \\$
 We have not been given enough data to determine the global sales margin as we have not been given any price value for the costs of production (only how they compare to the book Angela's Ashes printing costs which also does not have a price value). Therefore, we cannot say what the answer is.$\\$

$\textbf{Question 7} \\$
We need to divide the turnover by the number of shares for each company. The solution will be the company which produces the highest value. Letting $T_{company}$ denote the turnover per share for the specified company, an example of the calculation for the company RMP is:
\begin{align*}
T_{RMP} &= \frac{16.2}{0.25}\\
&= 64.80.
\end{align*}
Applying the same formula for the remaining companies we get (up to two decimal places): $T_{WT}=21$, $T_{WJ}=533.33$, $T_{BNT}=8.65$ and $T_{MR}=85.45$.

The highest value occurs with the company WJ. $\\$

$\textbf{Question 8} \\$
We need to divide the turnover by the assets for each company, and find the company with the lowest value. Letting $A_{company}$ denote the turnover per assets ratio for the specified company, an example of the calculation for the company RMP is:
\begin{align*}
A_{RMP} &= \frac{16.2}{25.4}\\
&= 0.6377...
\end{align*}
Applying the same formula for the remaining companies, we get (to 3 decimal places):
$A_{WT}=0.269$, $A_{WJ}=4.375$, $A_{BNT}=0.147$ and $A_{MR}=0.47$. $\\$

The lowest value occurs with the company BNT. $\\$

$\textbf{Question 9} \\$
Given a box of Beta costs £324 in year 1, to determine the cost of the box of Beta in year 4 we must determine year 4's inflation index as a fraction of year 1's inflation index and multiply this by the initial cost of the box of Beta. That is, the cost of the box of Beta in year 4, which we will denote by $B_4$, is:
\begin{align*}
B_4 &= \frac{99}{100} \times 324\\
&= 320.76.
\end{align*}
So the answer is £320.76. $\\$

$\textbf{Question 10} \\$
Rebasing the inflation index for Gamma to year 3, the inflation index there needs to be $100\%$. Given the inflation index for Gamma at year 3 is $94\%$, this is achieved by dividing all inflation indices by $94\%$, and multiplying by 100 (needed to get the inflation indices into percentage form again).
Doing this to the inflation index for Gamma in year 4 we find the new inflation index to be
$$\frac{91}{94} \times 100 = 96.8...$$
To the nearest whole number, the new inflation index in year 4 is $97\%$. $\\$

$\textbf{Question 11} \\$
$\textbf{Note/Comment:}$ You said "calculate each period's percentage decrease" in solutions but then you calculated their increase or change. A negative decrease would be an increase. $\textbf{ End of Note/Comment}$ $\\$

We need to calculate each period's percentage change, and determine which has the largest decrease. Letting $C_{X-Y}$ denote the percentage change between year X and year Y, we get:
\begin{align*}
C_{1-2} &= \bigg(\frac{97}{100} \times 100 \bigg) - 100\\
&= -3\%\\
C_{2-3} &= \bigg(\frac{94}{97} \times 100 \bigg) - 100\\
&= -3.09\%\\
C_{3-4} &= \bigg(\frac{91}{94} \times 100 \bigg) - 100\\
&= -3.19\%
\end{align*}
The largest percentage decrease occurred between year 3 and year 4 with a percentage decrease of $3.19\%$. $\\$

$\textbf{Question 12} \\$
We have been given no information in regards to the inflation index at Q5 and thus we cannot say. The value for the inflation index at the end of Q4 is the equivalent of the inflation index at the start of Q5, which we require to calculate the answer, but have not been given. $\\$

$\textbf{Question 13} \\$
$\textbf{Note/Comment:}$ Super confusing data, didn't really know what the numbers meant. Took me reading the solutions to understand what was going on. Explanation/detail wouldn't hurt.  $\textbf{ End of Note/Comment}$ $\\$

We are going to express the price of each part in terms of part A. Part A costs twice as much as part B, so $A=2B$ and thus $B= \frac{1}{2}A$. Part C costs three times as much as part A so $C=3A$. And lastly, part D costs twice as much as part A, so $D=2A$. $\\$

Now we will calculate the price of each product in terms of the price of part A, which we will denote $P_x$ where x indicates which product (i.e. x=A,B,C or D). So:
\begin{align*}
P_A &= 13A + 15B + 0 C + 12D\\
&= 13A + \frac{15}{2}A + 24A\\
&=44.5A,\\
P_B &= 22A + 12B + 47 C + 11D\\
&= 22A + 6A + 141A + 22A\\
&=191A,\\
P_C &= 22A + 15B + 8 C + 4D\\
&= 22A + \frac{15}{2}A + 24A + 8A\\
&=61.5A,\\
P_D &= 35A + 17B + 5 C + 7D\\
&= 35A + \frac{17}{2}A + 15A + 14A\\
&=72.5A.
\end{align*}

From this it is clear that product B is the most expensive. The second most expensive is product D, followed by product C and then lastly product A. The answer is product D. $\\$

$\textbf{Question 14} \\$
If the ABC sold 20 of product A at the price of 10 times part C, and if we let $C$ denote the price of part C, we get a revenue of $20 \times 10 \times C = 200C$. Given C costs 3 times as much as part A we have $C=3A$, and thus $200C=200 \times (3A) = 600A$. So for the given price, we could buy 600 of part A with the revenue made from product A. $\\$

Note that the fact that it says "Product A" is irrelevant to the question and is included as a decoy! $\\$

$\textbf{Question 15} \\$
The answer is cannot say. We are not given any information that will aid in calculating sales values of each product. We are only given information on parts costs and number of parts, which is not enough information. $\\$

$\textbf{Question 16} \\$
If ABC had parts for 60 of product C and 60 of product D, then we have 
\begin{align*}
22 \times 60 + 35 \times 60 &= 3420\text{ of part A available}\\
15 \times 60 + 17 \times 60 &= 1920\text{ of part B available}\\
8 \times 60 + 5 \times 60 &= 780\text{ of part C available}\\
4 \times 60 + 7 \times 60 &= 660\text{ of part D available}
\end{align*}

Now to determine how much product B we could make with these parts, we must divide each type of part (i.e. part A or B), by the number of the same type of part required for product B. The minimum value rounded down to the nearest integer will be the maximum amount of product B that can be produced. So:
\begin{align*}
\text{Part A }&: \frac{3420}{22} = 155.45...\\
\text{Part B }&: \frac{1920}{12} = 160\\
\text{Part C }&: \, \, \frac{780}{47} =16.59...\\
\text{Part D }&: \, \, \frac{660}{11} = 60.
\end{align*}

There are only enough of part C to make 16 of Product B, so this is the most that can be made. $\\$


$\textbf{Question 17} \\$
Taking initial data to be the £45.92 bag of food, the inflation over 3 quarters will be compound, and the inflation of food in Q2 is $-2\%$, Q3 is $-1 \%$ and Q4 is $0\%$ as given by the graph. The cost of the food at the end Q4, $F_4$, is thus given by:
\begin{align*}
F_4 &= \bigg( \bigg( 45.92 \times \frac{100-2}{100} \bigg) \times \frac{100-1}{100} \bigg) \times \frac{100+0}{100}\\
&= 44.55...
\end{align*}
So the cost of the food at the end of Q4 is £44.55. $\\$

$\textbf{Question 18} \\$
$\textbf{Note/Comment:}$ The sentence begins wrong. 'If inflation the index...' should say 'If the inflation index...'$\textbf{ End of Note/Comment}$ $\\$

Basing the inflation index at the start of Q5, the initial data at Q5 will be the standard base of $100\%$. Now given we are going backwards instead of forwards, as opposed to multiplying the initial data by the inflation ratio, we must divide it by it instead. More explicitly, letting $I_1$ denote the inflation index at the end of Q1, we know the inflation index at the beginning of Q5 is $100\%$, and so from the standard formula for calculating the compounded inflation index we get:
\begin{align*}
\bigg( \bigg( I_1 \times \frac{100+2}{100} \bigg) \times \frac{100+3}{100} \bigg) \times \frac{100-1}{100} = 100.
\end{align*}
Rearranging this in terms of $I_1$ we get the required formula:
\begin{align*}
I_1 &= \bigg( \bigg( 100 \times \frac{100}{99} \bigg) \times \frac{100}{103} \bigg) \times \frac{100}{102}\\
&= 96.145...
\end{align*}
Thus the inflation index at the end of Q1 (to two decimal places) is $96.15\%$. $\\$

$\textbf{Question 19} \\$
Th graph shows the inflation over each quarter. In order for prices to remain constant in a given quarter we require that the graph be constant with a value of zero inflation over the whole of any given quarter. As is clear from the graph of alcohol, this is never the case and so the solution is none of these. $\\$

The period of Q2-Q3 shows a constant inflation, however the value of inflation is $1\%$ over this period, which does not result in constant prices. $\\$

$\textbf{Question 20} \\$
In Q4, alcohol and books experienced a positive inflation, and thus a price increase. Food experiences a $0\%$ inflation and so neither an increase in price nor a decrease. Education on the other hand experiences a deflation as given by its negative inflation of $-1\%$. This means that the price of education decreases. Given education is the only product which decreases, it is the product which experience the greatest decrease in price, and is thus the correct answer. $\\$

$\textbf{Question 21} \\$
The average rate of inflation in Q2 is given by the sum of each products inflation rate in Q2 divided by the number of products there are. Thus:
\begin{align*}
\text{Average} &= \frac{2+1+1-2}{4}\\
&= 0.5.
\end{align*}
The average inflation is $0.5\%$. This is not on the list of solutions and so the answer is none of these.

\end{document}
